\documentclass[spanish]{mathnotes}

% Title page
\title{Métodos numéricos para EDP}
\subtitle{Comparación de diversos métodos aplicados a la ecuación del calor}
\author{Guillermo Ruiz Álvarez}
\date{\today}
\university{Universidad Autónoma de Madrid}

\begin{document}
	\makepre
	
	\section{Definición del problema}
	Consideramos el siguiente problema para la ecuación del calor
	\begin{equation*}
	\left\{
		\begin{array}{l l}
			u_t = u_{xx} & (x,t)\in(0,1)\times (0,t_F)\\
			u(0,t) = u(1,t) = 0 & t>0\\
			u(x,0) = u_0(x) = x(1-x) & 0\le x\le 1
		\end{array}
	\right.
	\end{equation*}
	
	Vamos a obtener aproximaciones a la solución de este problema utilizando el $\theta$-método para los siguientes valores de $\theta$:
	\begin{itemize}
		\item $\theta = 0$: método explícito.
		\item $\theta = 1$: método implícito.
		\item $\theta = \frac{1}{2}$: método de Crank-Nicolson.
	\end{itemize}
	
	En todo lo que sigue se utilizará la notación y los datos que siguen:
	\begin{center}
			\framebox{$t_F = 0.6$}\hspace{1em}
			\framebox{$\nu = \frac{\Delta t}{(\Delta x)^2}$}\hspace{1em}
			\framebox{$\mu = \frac{\Delta t}{\Delta x}$}
	\end{center}
	
	\section{Programación del $\theta$-método}
	El $\theta$-método tiene la forma siguiente (con \framebox{$j =  0,1, \hdots, J$} y \framebox{$n \in \left[0, t_F\right]$}): 
	$$\frac{U_j^{n+1}-U_j^n}{\Delta t} = 
	\theta \left[\frac{U_{j+1}^{n+1}-2U_{j}^{n+1}+U_{j-1}^{n+1}}{(\Delta x)^2}\right] +
	(1-\theta) \left[ \frac{U_{j+1}^n-2U_j^n+U_{j-1}^n}{(\Delta x)^2}\right]$$
	
	Matricialmente, el método tiene la expresión:
	$$AU^{n+1} = BU^n$$
	
	Y se basa en resolver la ecuación matricial y obtener:
	$$U^{n+1} = A^{-1} B U^n$$
	
	Donde:
	\begin{equation*}
		\begin{array}{l c r}
				U^n = \begin{bmatrix}
				U_1^{n}\\
				U_2^{n}\\
				\vdots\\
				U_{J-1}^{n}\\
				\end{bmatrix}
				&
				U^{n+1} = \begin{bmatrix}
				U_1^{n+1}\\
				U_2^{n+1}\\
				\vdots\\
				U_{J-1}^{n+1}\\
				\end{bmatrix}
				&
				A =
				\begin{bmatrix}
				1+2\nu\theta & -\nu\theta &        & \\
				-\nu\theta   & \ddots     & \ddots & \\
				& \ddots     & \ddots     & -\nu\theta\\
				&            & -\nu\theta & 1+2\nu\theta\\
				\end{bmatrix}
		\end{array}
	\end{equation*}
	\begin{equation*}
	B = 
	\begin{bmatrix}
	1-(1-\theta)2\nu         & (1-\theta)\nu&        & \\
	(1-\theta)\nu            & \ddots       & \ddots & \\
	& \ddots & \ddots        & (1-\theta)\nu\\
	&        & (1-\theta)\nu & 1-(1-\theta)2\nu\\
	\end{bmatrix}
	\end{equation*}
		
	A continuación se muestra el código en \textsc{MatLab} del $\theta$-método para la ecuación del calor con las condiciones iniciales y de contorno descritas.
	
	
	\lstset{style=matlabStyle}
	\lstinputlisting{../src/thetamet.m}
	
	\newpage
	Para crear las matrices $A$ y $B$ se ha programado una función que acepta como parámetros el tamaño de la matriz y los valores de las tres diagonales y devuelve la matriz tridiagonal optimizada.
	\lstset{style=matlabStyle}
	\lstinputlisting{../src/tridiag.m}
	
	\section{Programación de la solución real}
	La solución real del problema es:
	$$u(x,t) = \sum_{m=1}^\infty B_m e^{-(m\pi)^2 t} sin(m\pi x)$$ 
	
	Los coeficientes $B_m$ tienen la expresión: $$B_m = 2\int_0^1 u_0(x) sin(m\pi x)$$
	
	Realizando la integral por partes se obtiene:
	$$B_m = = \frac{4\left[ (-1)^{m+1}+1\right]}{\pi^3m^3}$$
	
	Para programar el método se ha escrito una función que realiza la suma y la trunca cuando llega a cierto número de iteraciones. El código de dicha función se muestra a continuación.
	
	\lstset{style=matlabStyle}
	\lstinputlisting{../src/heat_sol.m}
	
	\section{Ejercicio 1}
	A continuación se muestra un gráfico en escala doblemente logarítmica representando $J$ frente al error en norma infinito en tiempo $t_F = 0.6$ usando distintos valores de $\nu$ y $\mu$ para los tres métodos.
	
	\subsection{Método explícito}
	
	
	\subsection{Método implícito}
	
	
	\subsection{Método de Crank-Nicolson}
		
	\section{Ejercicio 2}
	A continuación se muestra un gráfico en escala doblemente logarítmica representando $t_{cpu}$ frente al error en norma infinito en tiempo $t_F = 0.6$ usando distintos valores de $\nu$ y $\mu$ para los tres métodos.
	
	\subsection{Método explícito}
	
	
	\subsection{Método implícito}
	
	
	\subsection{Método de Crank-Nicolson}
	
	
\end{document}